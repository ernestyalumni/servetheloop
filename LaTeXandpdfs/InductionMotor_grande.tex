% file: nductionMotor.tex
% Induction Motors, especially Linear Induction Motors (LIM), in unconventional ``grande'' format; fitting a widescreen format
% 
% github        : ernestyalumni
% linkedin      : ernestyalumni 
%
% This code is open-source, governed by the Creative Common license.  Use of this code is governed by the Caltech Honor Code: ``No member of the Caltech community shall take unfair advantage of any other member of the Caltech community.'' 
% 

\documentclass[10pt]{amsart}
\pdfoutput=1
\usepackage{mathtools,amssymb,lipsum,caption}

\usepackage{graphicx}
\usepackage{hyperref}
\usepackage[utf8]{inputenc}
\usepackage{listings}
\usepackage[table]{xcolor}
\usepackage{pdfpages}
%\usepackage[version=3]{mhchem}
\usepackage{mhchem}

\usepackage{tikz}
\usetikzlibrary{matrix,arrows}

\usepackage{multicol}

\hypersetup{colorlinks=true,citecolor=[rgb]{0,0.4,0}}

\oddsidemargin=15pt
\evensidemargin=5pt
\hoffset-45pt
\voffset-55pt
\topmargin=-4pt
\headsep=5pt
\textwidth=1120pt
\textheight=595pt
\paperwidth=1200pt
\paperheight=700pt
\footskip=40pt








\newtheorem{theorem}{Theorem}
\newtheorem{corollary}{Corollary}
%\newtheorem*{main}{Main Theorem}
\newtheorem{lemma}{Lemma}
\newtheorem{proposition}{Proposition}

\newtheorem{definition}{Definition}
\newtheorem{remark}{Remark}

\newenvironment{claim}[1]{\par\noindent\underline{Claim:}\space#1}{}
\newenvironment{claimproof}[1]{\par\noindent\underline{Proof:}\space#1}{\hfill $\blacksquare$}

%This defines a new command \questionhead which takes one argument and
%prints out Question #. with some space.
\newcommand{\questionhead}[1]
  {\bigskip\bigskip
   \noindent{\small\bf Question #1.}
   \bigskip}

\newcommand{\problemhead}[1]
  {
   \noindent{\small\bf Problem #1.}
   }

\newcommand{\exercisehead}[1]
  { \smallskip
   \noindent{\small\bf Exercise #1.}
  }

\newcommand{\solutionhead}[1]
  {
   \noindent{\small\bf Solution #1.}
   }


  \title{% Induction Motors
  \large including Linear Induction Motors (LIM), single-sided LIMs (SLIM)}

\author{Ernest Yeung \href{mailto:ernestyalumni@gmail.com}{ernestyalumni@gmail.com}}
\date{6 mars 2017}
\keywords{Induction motor, Linear Induction Motor, Electromagnetism, Electrodynamics}
\begin{document}

\definecolor{darkgreen}{rgb}{0,0.4,0}
\lstset{language=Python,
 frame=bottomline,
 basicstyle=\scriptsize,
 identifierstyle=\color{blue},
 keywordstyle=\bfseries,
 commentstyle=\color{darkgreen},
 stringstyle=\color{red},
 }
%\lstlistoflistings

\maketitle


\noindent gmail        : ernestyalumni \\
linkedin     : ernestyalumni \\
twitter      : ernestyalumni \\

\begin{multicols*}{2}

  
\setcounter{tocdepth}{1}
\tableofcontents



\begin{abstract}
Everything about Linear Induction Motors (LIM).  

I start from a theoretical and/or mathematical physicist's ``first principles'' perspective.  

Be aware that part of this document is to function as a ``dump'' - I dump all unorganized notes, direct quotes, and miscellaneous here without regard to organizing it in any cogent manner.  

\end{abstract}

\part{}

apply Helmholtz Eq., Eq. (1.34) of Boldea


\end{multicols*}

\begin{thebibliography}{9}

\bibitem{GiEa1985}  
J.F.Gieras and A.R.Eastham, “Performance calculation for single-sided linear induction motors with a double-layer reaction rail
under constant current excitation,” IEEE Transactions on Magnetics,
vol. 22, pp. 54–62, January 1986.


\bibitem{Bold2013}
Ion Boldea (Author), \textbf{Linear Electric Machines, Drives, and MAGLEVs Handbook}. 1st Edition.  CRC Press; 1 edition (February 7, 2013).  ISBN-13: 978-1439845141


\bibitem{}


\end{thebibliography}

\end{document}
