% file: servetheloop_dump_grande.tex
% rLoop notes "dump" 
% 
% github        : ernestyalumni
% linkedin      : ernestyalumni 
%
% This code is open-source, governed by the Creative Common license.  Use of this code is governed by the Caltech Honor Code: ``No member of the Caltech community shall take unfair advantage of any other member of the Caltech community.'' 
% 
\documentclass[10pt]{amsart}
\pdfoutput=1
\usepackage{mathtools,amssymb,lipsum,caption}

\usepackage{graphicx}
\usepackage{hyperref}
\usepackage[utf8]{inputenc}
\usepackage{listings}
\usepackage[table]{xcolor}
\usepackage{pdfpages}
%\usepackage[version=3]{mhchem}
\usepackage{mhchem}

\usepackage{tikz}
\usetikzlibrary{matrix,arrows}

\usepackage{multicol}

\hypersetup{colorlinks=true,citecolor=[rgb]{0,0.4,0}}

\oddsidemargin=15pt
\evensidemargin=5pt
\hoffset-45pt
\voffset-55pt
\topmargin=-4pt
\headsep=5pt
\textwidth=1120pt
\textheight=595pt
\paperwidth=1200pt
\paperheight=700pt
\footskip=40pt


\newtheorem{theorem}{Theorem}
\newtheorem{corollary}{Corollary}
%\newtheorem*{main}{Main Theorem}
\newtheorem{lemma}{Lemma}
\newtheorem{proposition}{Proposition}

\newtheorem{definition}{Definition}
\newtheorem{remark}{Remark}

\newenvironment{claim}[1]{\par\noindent\underline{Claim:}\space#1}{}
\newenvironment{claimproof}[1]{\par\noindent\underline{Proof:}\space#1}{\hfill $\blacksquare$}

%This defines a new command \questionhead which takes one argument and
%prints out Question #. with some space.
\newcommand{\questionhead}[1]
  {\bigskip\bigskip
   \noindent{\small\bf Question #1.}
   \bigskip}

\newcommand{\problemhead}[1]
  {
   \noindent{\small\bf Problem #1.}
   }

\newcommand{\exercisehead}[1]
  { \smallskip
   \noindent{\small\bf Exercise #1.}
  }

\newcommand{\solutionhead}[1]
  {
   \noindent{\small\bf Solution #1.}
   }


  \title{%servetheloop dump 
  \large $\#$servetheloop dump (for the rLoop) }

\author{Ernest Yeung \href{mailto:ernestyalumni@gmail.com}{ernestyalumni@gmail.com}}
\date{27 mai 2017}
\keywords{Eddy current brakes}
\begin{document}

\definecolor{darkgreen}{rgb}{0,0.4,0}
\lstset{language=Python,
 frame=bottomline,
 basicstyle=\scriptsize,
 identifierstyle=\color{blue},
 keywordstyle=\bfseries,
 commentstyle=\color{darkgreen},
 stringstyle=\color{red},
 }
%\lstlistoflistings

\maketitle


\noindent gmail        : ernestyalumni \\
linkedin     : ernestyalumni \\
twitter      : ernestyalumni \\

\setcounter{tocdepth}{1}
\tableofcontents


\begin{multicols*}{2}

  



\begin{abstract}
servetheloop notes "dump" - I dump all my notes, including things that I tried but are wrong, here.  

\end{abstract}

\part{Eddy Currents, Eddy Current Braking}

\section{Eddy Currents}

\emph{Keywords:} Eddy currents;

cf. Smythe (1968), Ch. X (his Ch. 10) \cite{Smyt1968}

Assume Maxwell's "displacement current" is negligible; this is ok if frequencies are such that wavelength $\lambda$ large compared to dimensions of apparatus $L$.  $\lambda \gg L$ or $\frac{c}{\nu} \gg L$.  

I will write down the "vector calculus" formulation of electrodynamics, along side Maxwell's equations, or electrodynamics, over spacetime manifold $M$.  The latter formulation should specialize to the "vector calculus" formulation.  

From 
\[
\text{curl} \mathbf{E} = -\frac{d\mathbf{B}}{dt} \, (SI) \qquad \, \text{curl}\mathbf{E} = -\frac{1}{c} \frac{ \partial \mathbf{B}}{ \partial t} \, (cgs) \text{ or } \frac{ \partial B}{ \partial t} + \mathbf{d}E = 0 
\]
Suppose $B = \text{curl} A$ or $B=\mathbf{d}A$ (EY 20170528: is this where the assumption above about $\lambda \gg L$ comes in?), then
\[
\begin{gathered}
	-\frac{\partial B}{ \partial t} = \mathbf{d}E \xrightarrow{ \int_S } \int_S \mathbf{d}E = \int_{ \partial S} E = -\int_S \frac{ \partial B}{ \partial t} \xrightarrow{ B = \mathbf{d}A } -\int_S \frac{ \partial }{ \partial t} \mathbf{d}A \xrightarrow{ \text{ flat space } } \int_S \mathbf{d}E = -\int_S \mathbf{d} \frac{ \partial A}{ \partial t}	
\end{gathered}
\]
and so
\begin{equation}
	\mathbf{E} = \frac{-\partial \mathbf{A}}{ \partial t}
\end{equation}
up to gauge transformation, if $B = \mathbf{d}\mathbf{A} = \text{curl}\mathbf{A}$

Since this $\mathbf{E}$ field is formed in a conductor, Ohm's law applies.  Let's review Ohm's law.  Smythe (1968) refers to its 6.02 Ohm's Law - Resistivity section \cite{Smyt1968}.  Indeed, in a lab, the definition of resistance can be defined as this ratio:
\begin{equation}
R_{AB} := \frac{ -\int_A^B \mathbf{E} }{ I_{AB} } = \frac{ V_A- V_B}{ I_{AB} } = \frac{ \varepsilon_{AB} }{ I_{AB} }
\end{equation}
Moving along, the right way to think about resistivity $\rho$ is to consider conductivity.  

Assume current density is linear to $\mathbf{E}$ field (as $\mathbf{E}$ field pushes charges along).  This linear response is reasonable.  

Also, assume current density $\mathbf{J}$ is uniform over infinitesimal surface area $dA$ (i.e. surface $S$).  

Define 
\begin{equation}
\sigma \equiv \text{ \textbf{ conductivity } }
\end{equation}

Then the empirical relation/equation that underpins \emph{Ohm's law} is 
\begin{equation}
\mathbf{J} = \sigma \mathbf{E}
\end{equation}
and define \emph{resistivity} from there:
\begin{equation}
\sigma := \frac{1}{\rho}
\end{equation}
where $\rho $ is the \emph{resistivity}. 

Thus
\[
\begin{gathered}
	\sigma \int_A^B -\mathbf{E} = \sigma V_{AB} = \int_A^B \mathbf{J} = I_{AB} \frac{l}{A} \\
\frac{1}{\rho} R = \frac{l}{A} \text{ or } \boxed{ R = \rho \frac{l}{A} }
\end{gathered}
\]


From Maxwell's Equations, 
\begin{equation}
	\mathbf{\delta}(B - 4\pi \mathbf{M} ) = \frac{4\pi }{c} J_{\text{free}} 
\end{equation}	

If $B=H+4\pi M = (1+4\pi \chi_m) H = \mu H$, 
then
\begin{equation}
\mathbf{\delta}H = \mathbf{\delta} \frac{B}{\mu} = \frac{ 4\pi }{c} J_{\text{free} }\text{ or } \mathbf{\delta} B = \mu \frac{4\pi }{c} J_{\text{free}} \Longleftrightarrow \text{curl} B = \mu J_{\text{free}} \qquad \, (SI)
\end{equation}
and if $B=\mathbf{d}A$ and $\mathbf{d} \mathbf{\delta}A =0$.  

%\subsubsection{Eddy Currents}

I build upon the physical setup proposed by Jackson (1998) \cite{Jack1998} in Section 5.18 "Quasi-Static Magnetic Fields in Conductors; Eddy Currents; Magnetic Diffusion."   

For a system (with characteristic) length $L$, $L$ being small, \\
compared to electromagnetic wavelength associated with dominant time scale of problem $T$, 
\[
\begin{gathered}
	f := \frac{1}{T} ; \quad \, \omega = 2\pi f ; \quad \, \omega \lambda = c \Longrightarrow \lambda = \frac{c}{ \omega } = \frac{c}{ 2\pi f } = \frac{Tc}{2\pi }  \\
\frac{L}{\lambda} = \frac{LTc}{2\pi } \gg 1
\end{gathered}
\]
From Maxwell's equations, in particular, Faraday's Law, and in its integral form (over 2-dim. \emph{closed} surface $S$), 
\begin{equation}
\begin{gathered}
\mathbf{d}E + \frac{ \partial }{ \partial t } B = 0 \text{ or } -\mathbf{d}E =\frac{ \partial B}{ \partial t} \xrightarrow{ \int_S } \int_S \frac{ \partial B}{ \partial t} = -\int_S \mathbf{d}E = -\int_{\partial S} E
\end{gathered}
\end{equation}
So on $S$, changing magnetic flux $\int \frac{ \partial B}{ \partial t}$ results in $E$ field, circulating around boundary of $S$, $\partial S$.  

We know that in a conductor, free conducting electrons get pushed around by $E$ fields, result in a current density $J$.  

$J$ is related to $E$, \emph{empirically} (by Ohm's Law)
\[
J = \sigma E
\]
where $\sigma$ is the resistivity.  

Then use the force law on this induced current $J$ from the $B$ field set up:
\[
F_{\text{net}} = \frac{1}{c} \int_S J\times B dA
\]
By working through the right-hand rule, $F_{\text{net}}$ the force on those currents induced in the conductor due to the $B$ that's there, is in the direction to help oppose changing (increasing or decreasing $\frac{\partial B}{ \partial t}$).  

To find $B$, suppose $B=dA$, i.e. $B\in H^2_{\text{deRham}}(M)$, i.e. $B=\text{curl}A$.  

For sure, 
\[
\mathbf{\delta} (B-4\pi c \mathbf{M}) = 4\pi J \Longleftrightarrow \text{curl}(B-4\pi c \mathbf{M} ) = \text{curl} H = 4\pi J
\]
Be warned now that the relation $B=\mu H$ may not be valid on all domains of interest; $\mu$ could even be a tensor! (e.g. $B_{ij} = \mu^{kl}_{ij} H_{kl}$).  However, both Jackson (1998) \cite{Jack1998} in Sec. 5.18 Quasi-Static Magnetic Fields in Conductors; Eddy Currents; Magnetic Diffusion, pp. 219, and Smythe (1968), Ch. X (his Ch. 10), pp. 368 \cite{Smyt1968}, continues on \emph{as if} this relation is linear: $B=\mu H$.  

Nevertheless, as we want to find $B$ by finding its "vector potential" $A$, we obtain a diffusion equation: 
\begin{equation}\label{Eq:EddyCurrentsAdiffusion}
\begin{gathered}
	- \mathbf{\delta}B = \mathbf{*d*d}A = (-1) \mathbf{\delta d} A = (-1)( \Delta - \mathbf{d\delta} ) A \xrightarrow{ \mathbf{d\delta} A = 0 } - \Delta A = \\
	= 4\pi \mu J = 4\pi \mu \sigma E = 4\pi \mu \sigma \left( -\frac{ \partial A}{ \partial t} \right) \\
\Longrightarrow \boxed{ \Delta A = 4\pi \mu \sigma \frac{ \partial A}{ \partial t } }
\end{gathered}
\end{equation}
where in the first 2 steps (equalities), $- \mathbf{\delta}B = \mathbf{*d*d}A = (-1) \mathbf{\delta d} A$ it's interesting to note that the codifferential $\mathbf{\delta}$ for the 2 form $B$ had to be written out explicitly, and then the codifferential for the 1-form $A$ is \emph{different} from the $\mathbf{\delta}$ for $B$ by a(n important) factor of $(-1)$; where $\mathbf{d\delta} A=0$ must be satisfied by the form $A$ takes; and where, since $B=\mathbf{d}A$,
\begin{equation}
\begin{gathered}
	\mathbf{d}E + \frac{ \partial B}{ \partial t} = \mathbf{d} E + \frac{ \partial }{ \partial t} \mathbf{d} A = \mathbf{d} \left( E+ \frac{ \partial A}{ \partial t} \right) = 0 \Longrightarrow E = -\frac{ \partial A}{ \partial t} + \text{grad}\Phi \xrightarrow{ \Phi = \text{ constant } } E = -\frac{ \partial A}{ \partial t}
\end{gathered}
\end{equation}
whereas a choice of gauge for $E$ was chosen so that $\Phi=\text{constant}$ (and so a particular form for $E$ was chosen, amongst those in the \emph{same} equivalence class of $H^1_{\text{deRham}}(M)$.  

To ensure that the differential geometry formulation is in agreement with the practical vector calculus formulation, compare Eq. \ref{Eq:EddyCurrentsAdiffusion} with Eq. (5.160) of Jackson (1998) \cite{Jack1998} and Eq. (10) in Sec. 10.00 of Smythe (1968) \cite{Smyt1968}.  

To summarize what's going on, I think one should at least understand in one's head how Maxwell's Equations apply, (and I will try to write in SI here)
\begin{equation}
\boxed{
\begin{gathered}
\int_S \frac{ \partial \mathbf{B}}{ \partial t} dA = -\oint \mathbf{E}\cdot d\mathbf{s} \Longrightarrow \mathbf{J}=\sigma \mathbf{E} \Longrightarrow \mathbf{F}_{\text{net}} = \int_S \mathbf{J} \times \mathbf{B} dA  \\
\text{ find } \mathbf{B} = ? \qquad \, \text{ using form } \mathbf{B} = \nabla \times \mathbf{A}, \\
\nabla^2 \mathbf{A} = \mu \sigma \frac{ \partial \mathbf{A} }{ \partial t} \qquad \, (SI)
\end{gathered}
}
\end{equation}
where, a change in magnetic flux over a surface $S$ over the conductor, $\int_S \frac{ \partial \mathbf{B}}{ \partial t}dA$ induces a circulation of $E$ field around $S$, $-\oint \mathbf{E} \cdot d\mathbf{s}$, and this $E$ field is pushing around \emph{free conducting charges} according to Ohm's law, $\mathbf{J} = \sigma \mathbf{E}$, with $\sigma$ being the conductivity of the conducting material, and this current density $\mathbf{J}$ is then acted upon by the prevailing $B$ field, according to the usual force law.  To find $\mathbf{B}$, one can find $\mathbf{A}$ and \emph{try} to find $\mathbf{A}$ analytically.  

Keep in mind that for $\nabla^2 \mathbf{A} = \mu \sigma \frac{ \partial \mathbf{A}}{ \partial t}$, we had used, critically, the Maxwell equation $\mathbf{\nabla} \times \mathbf{H} = \mathbf{J}$, with $\mathbf{J}$ being the \emph{induced current of free conducting charges on the conductor}.  This $\mathbf{H}$ will contribute (through linear superposition) to the $\mathbf{B}$ that could already be there due to the permanent magnet.  

What can we measure quantitatively?
\begin{itemize}
\item Can we measure $\mathbf{J}$ inside (on) the conductor?
\item Can we separate magnetization $\mathbf{M}$ of material from $\mathbf{B}$, to obtain the actual $\mathbf{B}$ (and then use linear superposition, $\mathbf{B}_{\text{total}} = \mathbf{B}_{\text{permanent magnet}} + \mathbf{B}_{\text{induced currents}}$? 
\end{itemize}

Also, keep in mind the context that the conductor at hand is the long, almost semi-infinite rectangle of a conductor, aluminum sub-rail, specified by the SpaceX Hyperloop.  Force on that will cause an equal and opposite force on the pod, with its permanent magnets attached, and thus braking the pod.   

\part{System Theory, including Avionics, Embedded Systems}

%\part{System Theory}

cf. Hermann (1974) \cite{Herm1974}.  

For a given field $\mathbb{K} = \mathbb{R} \text{ or } \mathbb{C}$ (or some field, in general).  

Let $U,X,Y$ be either vector spaces (or more generally, $R$-modules, with $R=\mathbb{K}$), or manifolds.   Let
\[
\begin{aligned}
	& U \equiv \text{ input space } \\
	& Y \equiv \text{ output space } \\
	& X \equiv \text{ state space } \\
\end{aligned}
\]

Consider maps $f,g$, s.t. 
\begin{equation}
	\begin{gathered}
	\begin{aligned} 
& f: X\times U \to X \\
	& g : x \times U \to Y \end{aligned} \\
\begin{aligned}
	& \frac{dx}{dt} = f(x,u) \\ 
	& y(t) = g(x(t),u(t))  
\end{aligned}
\end{gathered}
\end{equation}
 
Ordered 5-tuple $\sigma = (U,X,Y,f,g)$ is a \textbf{system}.  

$t \mapsto (x(t),u(t),y(t)) \in X\times U \times Y =: $ \textbf{trajectory} of system, they're solutions to $\frac{dx}{dt} = f(x,u)$.  

\begin{definition}
	Input-output pair: $t\mapsto (u(t),y(t)) \in U\times Y$ if $\exists \, $ curve $t\mapsto x(t) \in X$ s.t. 
\begin{equation}
t\mapsto (x(t),u(t),y(t))
\end{equation}
\end{definition}

Choose input curve $t\mapsto u(t)$ so to achieve some desired output $t\mapsto y(t)$.  

If we have linear maps $f,g$ or at least "linearize" the system, then $f,g$ take forms 
\begin{equation}
\begin{gathered} 
\begin{aligned}
& f(x,u) = Ax+Bu \\ 
& g(x,u) = Cx+Du
\end{aligned}  \\
\begin{aligned}
	& A:X\to X \\ 
	& B: U\to X \\ 
	& C: X\to Y \\ 
	& D: U\to Y
\end{aligned}
\end{gathered}
\end{equation}

And so, for a (linear) system $(X,U,Y,A,B,C,D,)$, s.t. 
\[
\begin{aligned}
	& \frac{dx}{dt} = Ax(t) + Bu(t) \\ 
	& y(t) = Cx(t) + Du(t) 
\end{aligned}
\]
Discretize $t$ or replace differential with difference equations.  And so for  \\
$t\in $ additive semigroup $\mathbb{Z}^+$, 
\begin{equation}
\begin{gathered}
\begin{aligned}
	& t \mapsto u(t) \in \mathbb{Z}^+ \to U \\ 	
	& t \mapsto y(t) \int \mathbb{Z}^+ \to Y
\end{aligned} \\
	 \frac{du}{dt} \xrightarrow{ \text{ discretize } } \Delta u(t) = u(t+1) - u(t) \\
\begin{aligned}
	& \Delta u(t) = Ax(t) + Bu(t) \\ 
	& y(t) = Cx(t) + Du(t)
\end{aligned}
\end{gathered}
\end{equation}

Equivalence of systems with same $X,U,Y$.  

Let $GL(X) = \lbrace g | g:X \to X \rbrace$.  

Given $t \mapsto (x(t),u(t),y(t))$ s.t. it's a solution to 
\begin{equation}
\begin{aligned}
	& \frac{dx}{dt} = Ax(t) + Bu(t) \\ 
	& y(t) = Cx(t) + Du(t) 
\end{aligned} \qquad \, (3.2 of Hermann (1974))  
\end{equation}
cf. Hermann (1974) \cite{Herm1974}.  

Set 
\begin{equation}
	x_1(t) = gx(t) 
\end{equation}
Then
\[
\begin{gathered}
	\begin{gathered}
	\frac{dx_1(t)}{dt} = g\frac{dx}{dt} = g(Ax+Bu) = A_1x_1 + B_1u  \qquad \, \text{ and so }  \begin{aligned} & A_1 = gAg^{-1}  \\
& B_1= gB \end{aligned} 
\end{gathered}
\begin{gathered}
y(t) = Cg^{-1}x_1 + Du = C_1x_1 + D_1 u \qquad \, \text{ with } \begin{aligned} & C_1 = Cg^{-1} \\ 
& D_1 = D \end{aligned}
\end{gathered}
\end{gathered}
\]
\[
\Longrightarrow \sigma_1 = (U,X,Y,A_1 ,B_1, C_1,D_1) 
\]
$\sigma,\sigma_1$ automatically have the same input-output pairs.  

\begin{definition}
\begin{equation}	\begin{aligned} 
	& \sigma = (A,B,C,D) \\
	& \sigma_1 = (A_1,B_1,C_1,D_1) \end{aligned}
\end{equation}  2 linear systems with same state, input, output space $X,U,Y$, 
$\sigma,\sigma_1$ \textbf{ algebraically equivalent  } iff $\exists \, g \in GL(V)$ s.t. 
\begin{equation}
\begin{aligned}
& A_1 = gAg^{-1} \\ 
& B_1 = gB \\ 
& C_1 = Cg^{-1} \\ 
& D_1 = D
\end{aligned}
\end{equation}
\end{definition}
i.e. Let $\Sigma = \lbrace \sigma \rbrace =$ collection of linear systems $\sigma = (A,B,C,D)$, \\
i.e. $\Sigma = L(X,X) \times L(U,X) \times L(X,Y) \times L(U,Y)$.  

Let $GL(X) = \lbrace g|g:X\to X, \, \forall \, g, \, \exists g^{-1} \rbrace$.  

If $g\in GL(X)$ and 
\[
\sigma = (A,B,C,D) \in \Sigma
\]
then 
\[
g\sigma = (gAg^{-1}, gB,Cg, D)
\]
\begin{theorem}[3.2] 
	equivalence classes in 1-to-1 correspondence with \textbf{orbit space } $GL(X) \backslash \Sigma$.  
\end{theorem}

\subsubsection{cf. 4. Impulse response and Transfer Functions: Observatibility and Controllability of Hermann (1974) \cite{Herm1974}.  }

From Eq. (3.2) of Hermann (1974) \cite{Herm1974}, i.e. $\begin{aligned} & \quad \\ 
	& \dot{x} = Ax + Bu \\
	& y = Cx+Du \end{aligned}$, i.e. linear, time-invariant, "lumped parameter" systems.  

"weighting parameter" $=$ "impulse response".  

It's easy to solve
\[
\dot{x} = Ax + Bu 
\]
with solution
\[
x(t) = \int_0^t \exp{ (A(t-s) )} Bu(s)ds + \exp{(At)} x(0)
\]
since
\[
y = \int_0^t Ce^{A(t-s)} Bu(s)ds + Ce^{At} x(0) + Du
\]
i.e. by variation of parameters method.  

Consider 
\[
\begin{gathered}
	\dot{x}^1 - Ax^1 = 0 \\
x^1 = \exp{(At)}
\end{gathered}
\]
If $x_p(t) = c_1(t)x^1(t) = c_1 \exp{ (At) }$, 
\[
\begin{gathered}
\dot{x}_p = c_1e^{At} + c_1 Ae^{At} \\ 
\dot{x}_p = Ax_p - Bu = \dot{c}_1 + c_1 A - Ac_1 - Bu = 0  \\
\dot{c}_1 = Bu \text{ or } c_1 = \int_0^t ds Bu(s) \\
c_1(0) = x(0)   \qquad \, \text{ (EY : 20170602 I'm not sure if this is the right derivation) } 
\end{gathered}
\]
\begin{definition}
	Function 
\begin{equation}
	t\mapsto Ce^{At}B , \qquad \, R \to L(U,Y)
\end{equation}
is called \textbf{ impulse response } for system $\sigma = (A,B,C,D)$
\end{definition}

Its Laplace transform 
\begin{equation}
	s=C(A\cdot s)^{-1}B
\end{equation}
is called \textbf{frequency response} of system.  

\textbf{Remarks} A) \textbf{Resolvants}.  In functional analysis, operator-valued function $s\mapsto (A\cdot s)^{-1}$ is called resolvant of operator$A$.  

\begin{theorem}[4.1 Hermann (1974) \cite{Herm1974}]
Let 
\[
\begin{aligned}
&	\sigma = (U,X,Y,A,B,C,D) \\ 
&	\sigma_1 = (U,X_1,Y,A_1,B_1,C_1,D_1) \\ 
\end{aligned}
\]
Suppose $\sigma,\sigma_1$ equivalent, same input-output pairs.  Then
\begin{equation}
	C(A\cdot s)^{-1}B = C_1(A_1\cdot s)^{-1}B_1 \qquad \, s \in \mathbb{C}
\end{equation}
i.e. they have the same frequency responses.  Also $D=D_1$
\end{theorem}

\begin{proof}
Let $t \mapsto u(t)$ input curve, $x(0)$ initial state vector.  

$t\mapsto y(t)$ output curve determined by $(t\mapsto u(t), x(0))$ in system $\sigma$.  

Using Eq. (4.4) Hermann (1974) \cite{Herm1974}, 
\begin{equation}
\begin{gathered}
	y = \int_0^t Ce^{A(t-s) } Bu(s) ds + Ce^{At} x(0) + Du \\ 
y(t) = \int_0^t Ce^{A(t-s)} Bu(s) ds + Ce^{At } x(0) + Du(t) 
\end{gathered}
\end{equation}
By hypotheses, $\sigma,\sigma_1$ equivalent, so  \\

state vector $x_1(0) \in X$, s.t.
\[
(t\mapsto u(t), x_1(0)) \text{ determines } t\mapsto y(t), \text{ i.e. } 
\]
\[
y(t) = \int_0^t C_1 e^{A_1(t-s) } B_1u(s) ds + C_1 e^{A_1t } x_1(0) + D_1u(t)
\]

... 20170602 ...

\end{proof}

%\section{Embedded Systems}

\section{Finite State Machines}  

Lee and Seshia \cite{LeSe2017}


\begin{definition}[memoryless, memoryless system]  
Consider cont. time system $S: X\to Y$, where $X= A^{\mathbb{R}}$, $Y=B^{\mathbb{R}}$, 
\[
S:(\mathbb{R}\to A) \to (\mathbb{R} \to B) \text{ i.e. } S:L(\mathbb{R},A) \to L(\mathbb{R},B) 
\]
$A,B\in \textbf{Sets}$

$S$ \textbf{memoryless} if $\exists \, f :A\to B$ s.t. $\forall \, x$, 
\begin{equation}
(S(x))(t) = f(x(t)) \qquad \, \forall \, t \in \mathbb{R}
\end{equation}
i.e. output $(S(x))(t)$ depends \emph{only} on $x(t)$  
\end{definition}
cf. Sec. 2.3.2. Memoryless Systems of Lee and Seshia (2017) \cite{LeSe2017}

\begin{definition}[linear system; superposition]
system $S:X\to Y$ is \textbf{linear} if it obeys \textbf{superposition} property:
\begin{equation}
\begin{gathered}
	\forall \, x_1, x_2 \in X , \qquad \forall \, a,b, \in \mathbb{R} \\
S(ax_1 + bx_2) = aS(x_1) + bS(x_2)
\end{gathered}
\end{equation}
\end{definition}
cf. Sec. 2.3.3. Linearity and Time Invariance of Lee and Seshia (2017) \cite{LeSe2017}


\begin{definition}[delay]
\textbf{delay} $D_{\tau}:X\to Y$, $X,Y = C^1(\mathbb{R})$, so $D_{\tau}: C^1(\mathbb{R}) \to C^1(\mathbb{R})$  
\begin{equation}
	\forall \, x \in X, \, \forall \, t \in \mathbb{R}, \, (D_{\tau}(x))(t) = x(t-\tau)
\end{equation}
\end{definition}
cf. Sec. 2.3.2. Memoryless Systems of Lee and Seshia (2017) \cite{LeSe2017}

\begin{definition}[time invariant]
	system $S:X\to Y$ \textbf{time-invariant} if 
\begin{equation}
\begin{gathered}
\forall \, x \in X, \, \forall \, \tau \in \mathbb{R}, \\
	S(D_{\tau}(x)) = D_{\tau}(S(x))
\end{gathered}
\end{equation}
\end{definition}
Also, consider
\[
S(D_{\tau}(x))(t) = D_{\tau}(S(x))(t) = (S(x))(t-\tau)  \equiv y(t-\tau)
\]
e.g. Consider 
\[
\dot{\theta}_y(t) = \frac{1}{I_{yy}} \int_{-\infty}^t\tau_y(t')dt'  \qquad \, \text{ so } \tau_y \mapsto \dot{\theta}_y
\]
\[
\begin{gathered}
	S(D_{\tau}\tau_y) = \frac{1}{ I_{yy}} \int_{-\infty}^t \tau_y(t'-\tau) dt' \\
D_{\tau}S(\tau_y)(t) = \frac{1}{ I_{yy}}\int_{-\infty}^{t-\tau} \tau_y(t')dt' = \frac{1}{I_{yy}}\int_{-\infty}^t \tau_y(u-\tau) du \equiv \frac{1}{ I_{yy}} \int_{-\infty}^t \tau_y(t'-\tau) dt' = \frac{1}{I_{yy}} \int_{-\infty}^t (D_{\tau}\tau_y)(t')dt' = S(D_{\tau}\tau_y)(t)  \\
\begin{aligned}
& t'=t-\tau \\
& t' + \tau = t\\
& u=t' + \tau \end{aligned}
\end{gathered}
\]

\begin{definition}[stable, bounded-input bounded-output stable (BIBO stable or just stable)]
system $S:X\to Y$, $X,Y = C^1(\mathbb{R})$ \textbf{stable} if $\forall \, w \in X$, s.t. $\exists \, A \in \mathbb{R}$ s.t. $|w(t) | \leq A$, \, $\forall \, t\in \mathbb{R}$,  \\
$S(w) = v$, $\exists \, B\in \mathbb{R}$ s.t. $|S(w)(t)| = |v(t)| \leq B$, \, $\forall \, t\in \mathbb{R}$.  

\end{definition}

\subsubsection{Feedback Control: is it just an automorphism?}

Recall physics of the helicopter example:
\[
\begin{gathered}
	\dot{\theta}_y(t) = \dot{\theta}_y(0) + \frac{1}{I_{yy}} \int_0^t \tau_y(t')dt' \equiv (\text{int}(\dot{\theta}_y(0), I_{yy} ) )(\tau_y)(t), \text{ i.e. } \tau_y \xmapsto{ \text{int}(\dot{\theta}_y(0),I_{yy}) } \dot{\theta}_y
\end{gathered}
\]
Let $\psi$ be example of \emph{desired behavior}, input $\psi$ specifying desired angular velocity.  

Introduce error measurement $e\equiv \text{err}$:
\[
\begin{gathered}
	\text{err}:\mathbb{R} \times \mathbb{R} \to \mathbb{R} \\ 
\text{err}:\psi,\dot{\theta}_y \mapsto \psi- \dot{\theta}_y, \, \text{err}(\psi,\dot{\theta}_y)(t) = \text{err}(\psi(t),\dot{\theta}_y(t)) \equiv \text{err}(t) = \psi(t) - \dot{\theta}_y(t)
\end{gathered}
\]

\subsection{Discrete signals}

cf. 3.1 Discrete Systems of Lee and Seshia (2017) \cite{LeSe2017}

\begin{definition}[discrete signal]
\begin{equation}
\begin{aligned}
& e:\mathbb{R} \to \lbrace \text{ absent } \rbrace \cup X \equiv \lbrace \emptyset \rbrace \cup X \\ 
& e:t\mapsto e(t)
\end{aligned}
\end{equation}
\end{definition}

Let $T\subseteq \mathbb{R}$ s.t. 
\[
T:= \lbrace t\in \mathbb{R} | e(t) \neq \text{ absent } \equiv \emptyset 
\]
i.e. set of times when there's signal  

If $\exists \, $ 1-to-1 function $f:T\to \mathbb{N}$ s.t. $f$ \textbf{order-preserving}, i.e. $\forall \, t_1, t_2 \in T$, where $t_1\leq t_2$, then $f(t_1) \leq f(t_2)$.  

Example: Integration $I_i$: 
\[
\begin{gathered}
	\begin{aligned}
	& I_i: \mathbb{R}^{\mathbb{R}_+} \to \mathbb{R}^{\mathbb{R}_+} \equiv I_i : (\mathbb{R}_+ \to \mathbb{R}) \to (\mathbb{R}_+ \to \mathbb{R}) \\ 
	& y= I_i(x)
\end{aligned}
\end{gathered}
\]
where $i\equiv $ initial value of integration and $x,y$ are cont.time signals, i.e. $x,y \in C^1(\mathbb{R})$.  

e.g. If $i=0$, $\forall \, t \in \mathbb{R}_+$, $x(t) = 1$, then 
\[
y(t) = i+ \int_0^t x(\tau) d\tau = t
\]

Example: Counter $C_i$: 
\[
C_i: (\mathbb{R}_+ \to \text{Set})^P \to (\mathbb{R}_+ \to \lbrace \empty \rbrace \cup \mathbb{Z} )
\]
with $\text{Set } \in \textbf{Set}$ so $\text{Set}$ is a \emph{particular} set., \\
$P\in \textbf{Set}$, \\
e.g. $P\equiv $ set of input ports, e.g. $P=\lbrace \text{up}, \text{down} \rbrace$.  

$C_i:(\mathbb{R}_+ \to \lbrace \text{ absent, present } \rbrace )^P \to (\mathbb{R}_+ \to \lbrace \text{ absent } \rbrace \cup \mathbb{Z})$  

For particular time $t$, 

suppose actor has input ports $P=\lbrace p_1, \dots p_N \rbrace$, \\
$\forall \, p \in P$, set $V_P \in \mathbf{Set}$ denotes values that maybe received on port $p$, when input is present $\equiv$ \textbf{type} of port $p$.  

EY: At a reaction, 
\[
\begin{gathered}
	y: P \to V_P \cup \lbrace \emptyset \rbrace \\ 
	y: p \mapsto y(p) \text{ or } \emptyset 
\end{gathered}
\]

Assume absent at times $t$ where reaction doesn't occur.  
e.g. 
$Q = \lbrace \text{ count } \rbrace$ 

In summary, a \textbf{discrete signal} $x_i$, $\forall \, i = 1,2, \dots |P|$, $P\in \textbf{Sets}$, set of all input ports,  is
\begin{equation}
\boxed{ 
\begin{gathered}\begin{aligned}
& x_i: \mathbb{R} \to \lbrace \emptyset \rbrace \cup X_i \\ 
& x_i:t \mapsto x_i(t) 
\end{aligned} X\in \textbf{Sets} \text{ or } X = \mathbb{R}
\end{gathered}
}
\end{equation}
Let 
\begin{equation}
\boxed{ T_{x_i} = \lbrace t \in \mathbb{R} | x_i(t) \neq \emptyset \rbrace } 
\end{equation}
be the set of times when there's signal.  

Let 
\begin{equation}
\boxed{ \begin{aligned}
& p_i : \mathbb{R} \to \mathbb{Z}_2 = \lbrace \text{ present , abset } \rbrace \\ 
& p_i: t\mapsto p_i(t)  
\end{aligned} }
\end{equation}
$\forall \, i = 1,2, \dots |P | $ tell us if a signal is present or absent.   

Similarly for the output $y_j$: 













\section{Finite-State Machines}

cf. 3.3 Finite-State Machines of Lee and Seshia (2017) \cite{LeSe2017}

\textbf{Finite-state} machine; assume states $\equiv S \in \textbf{FiniteSets}$.  

\subsection{Transitions; edges, directed edges }

cf. 3.3.1 Transitions of Lee and Seshia (2017) \cite{LeSe2017}

\textbf{transition} - directed edge.  $\forall \, $ transition, label with "guard/action", i.e. label with a "guard" and an "action"

\textbf{guard}- determines whether transition maybe taken on reaction \\
\textbf{action} - specfies \emph{what outputs are produced on each reaction} \\

guard is a \textbf{predicate} (boolean-valued) s.t. true when transition should be taken  \\
e.g. example 3.4., FSM model for garage counter \\
states $\equiv S = \lbrace 0 ,1, \dots M \rbrace$  


\begin{tabular}{l l c}
	$\begin{aligned}
	\neg \\ 
	\sim \\ 
	! 
	\end{aligned}$ & negation, not &  $\neg A$ true, iff $A$ false  \\ \hline
	$\begin{aligned}
	\wedge \\ \& \
	\end{aligned}$ & logical conjunction, and & if $A$ and $B$ both true, $T$, $A \wedge B$ true.  Else, it's false.    \\ \hline 
	$\begin{aligned}
	& \vee \\ & + \\ & \parallel \end{aligned}$ & logical (inclusion) disjunction, or & if $A$ or $B$ (or both) true, $A \vee B$ false, if both false, $A\vee B$ false  
	 \\ \hline  
	$\begin{aligned}
	& \oplus \\ 
	& \veebar \end{aligned}$ & exclusive disjunction, xor & $A \oplus B$ true when either $A$ or $B$, but not both are true.    
\end{tabular}

\begin{tabular}{c c c c c c}
\hline
$A$ & $B$ & $\neg A$ & $ A\wedge B$ & $A \vee B$ & $A \oplus B$ \\ \hline
T & T & F & T & T & F  \\
T & F & F & F & T & T \\ 
F & T & T & F & T & T \\ 
F & F & T & F & F & F  
\end{tabular}


\textbf{action} (in reference to Lee, Seshia (2011) \cite{LeSe2017}), on a transition, action specifies resulting valuation on output ports when transition is taken.  

"When assigning a value to an output port, we use notation name $:=$ value to distinguish assignment from a predicate, which would be written name $=$ value.  

Let \\
ports $P_{\text{in}} = \lbrace p_1^{(\text{in})}, \dots p_{N_{\text{in}}}^{(\text{in})}  \rbrace$ \\
output ports $P_{\text{out}} = \lbrace p_1^{(\text{out})}, \dots p_{N_{\text{out}}}^{(\text{out})}  \rbrace$ \\

At a reaction, 
\begin{equation}
\begin{gathered}
\begin{aligned}
& i: P_{\text{in}} \to V_{P_{\text{in}}} \cup \lbrace \emptyset \rbrace \\
& i: p \mapsto i(p) 
\end{aligned} \\
\begin{aligned}
& o: P_{\text{out}} \to V_{P_{\text{out}}} \cup \lbrace \emptyset \rbrace \\ 
& o: p \mapsto o(p)	
\end{aligned}
\end{gathered}
\end{equation}
Given 
\[
i(p_1^{(\text{in})}, \dots p_{N_{\text{in}}}^{(\text{in}) } )  \in ( V_{P_{\text{in}}} \cup \lbrace \emptyset \rbrace )^{P_{\text{in}} } \times S
\]
where $S \equiv $ states of the system,  \\

\textbf{guard}, $g_{\text{guard}}$, 
\begin{equation}
\begin{aligned}
& g_{\text{guard}} : ( V_{P_{\text{in}}} \cup \lbrace \emptyset \rbrace )^{P_{\text{in}} } \times S \to \mathbb{Z}_2 = \lbrace \text{true}, \text{false} \rbrace \\ 
& g_{\text{guard}}( p_1^{(\text{in})}, \dots p_{N_{\text{in}}}^{(\text{in})} ) \in \lbrace \text{true}, \text{false} \rbrace 
\end{aligned}
\end{equation}

\textbf{action}  $a$, 
\begin{equation}
\begin{aligned}
& a : g_{\text{guard}}, e \mapsto o( p_1^{(\text{out})}, \dots p_{N_{\text{out}}}^{(\text{out})} )
\end{aligned}
\end{equation}
where  $e\in E$, $E \equiv $ set of all possible transitions.  

cf. Sec. 3.3.3. Update Functions of Lee, Seshia (2011)  \cite{LeSe2017} 

(States, Inputs, Outputs, update, initial State) $\equiv  (S,I, O, u,s_0)$,  

where $u: S\times I \to S \times O$, and update $\equiv $ transition function (synonymous)

Let sequence of reactions be indexed by $n \in \mathbb{N}$.  

Then 
\[
\begin{aligned}
& s:\mathbb{N} \to S \\ 
& s(0) = s_0
\end{aligned}
\]
$x:\mathbb{N} \to I$, $x\equiv $ input valuations \\
$y:\mathbb{N} \to O$, $y\equiv $ output valuation.  

state machine dynamics  
\[
(s(n+1),y(n)) = u(s(n),x(n))
\]

\[
\begin{gathered}
	\begin{aligned}
	& i: P \to \bigcup_{i=1}^{ P^{(\text{in})}_{N_{\text{in}} }} V_{ P_i^{(\text{in}) } } \bigcup \lbrace \emptyset \rbrace \\ 
	& i : p \mapsto i(p) \in V_P \cup \lbrace \emptyset \rbrace \equiv \text{ value of part $p$ } 
\end{aligned} \\
i\in I \subset \text{Hom}(\textbf{Set}, \textbf{Set})
\end{gathered}
\]



\subsubsection{Graph $(V,E)$ viewpoint}

Let $V=S \equiv $ states if $S\in \textbf{FiniteSet}$ (finite number of states), then for convenience, we can label, 1-to-1, $\forall \, s \in S$ with $\alpha = 0,1,\dots |S|$.  

$\forall \, s_{\alpha} S$, \\
$\exists \, \lbrace p_i^{(\alpha)} \rbrace_{i=1}^{ P^{(\alpha)} }$ input ports, $\lbrace ( x_i^{(\alpha)}, p_i^{(\alpha)} \rbrace | x_i^{(\alpha)} \in (\mathbb{R} \to X_i^{(\alpha)} ) , p_i^{(\alpha)} \in \lbrace \text{ present } , \emptyset \rbrace \rbrace_{i=1}^{ |P^{(\alpha)}| }$    \\
$\exists \, \lbrace q_j^{(\alpha)} \rbrace_{j=1}^{ Q^{(\alpha)} }$ output ports, $\lbrace ( y_j^{(\alpha)}, q_j^{(\alpha)} \rbrace | y_j^{(\alpha)} \in Y_j^{(\alpha)}  , q_j^{(\alpha)} \in \lbrace \text{ present } , \emptyset \rbrace \rbrace_{j=1}^{ |Q^{(\alpha)}| }$  

Consider $e\equiv (s_{\alpha},s_{\beta}) \in E$, $s_{\alpha}$ the initial vertex (state), $s_{\beta}$ terminal vertex (state).  

$\forall \, (s_{\alpha}, s_{\beta})$, associate a guard $g: (x_1^{(\alpha)} \dots x_{P^{(\alpha)}}^{(\alpha)} ) \mapsto \lbrace T,F \rbrace$ and $y^{(\alpha)} = (y_1^{(\alpha)} \dots y_{Q^{(\alpha)}}^{(\alpha)}$

If $\forall \, (x_1^{( \alpha)} \dots x_{P^{(\alpha)}}^{(\alpha)})$, $\exists \, $ at most 1 $(s_{\alpha}, s_{\beta} )$, system is deterministic.  


\subsection{Nondeterministic Finite-State Machine (FSM); Formal Model}  

cf. 3.5.1. Formal Model of Lee and Seshia (2017) \cite{LeSe2017} 

\textbf{nondeterministic FSM}  
\begin{equation}
	(\text{States, Inputs, Outputs, possibleUpdates, initialStates}) \equiv (S,I,O,u,I_0)  \equiv (S,X,Y,u,I_0)
\end{equation}
with update relation (or \textbf{transition relation}) $u$
\[
u:S\times I \to 2^{S\times O} 
\]
where powerset of $S\times O$, $2^{S\times O}$, is the set of possible (next state, output valuation) pairs, \\
$I_0 = $ set of initial states.  





Let $\Sigma = \lbrace \text{ apart, together } \rbrace = \lbrace \text{ apa, tog } \rbrace$  

\[
I_0 \ni \lbrace \begin{gathered} 
x_1(0) := x_{1,0} \\ 
x_2(0) := x_{2,0} \\
\dot{x}_1(0) = \dot{x}_2(0) = 0 \end{gathered}
\]
\[
(\text{apa},\text{tog}) , g= \text{ if } x_1(t) = x_2(t)
\]
action $\equiv$ set action, use momentum conservation: $m_1 \dot{x}_1 + m_2 \dot{x}_2 = (m_1+m_2) \dot{x} $ or 
\[
\dot{x}(t) = \frac{m_1 \dot{x}_1(t) + m_2\dot{x}_2(t) }{ m_1 + m_2 } 
\]

Let $s \equiv F_{\text{stick}} \in \mathbb{R}_+$, represent the stickiness of the 2 masses.  
\[
F_2-F_1 = -k_2 (x(t) - p_2) + k_1(x(t) - p_1) > F_{\text{stick}} \text{ or } -(k_2- k_1) x(t) + k_2 p_2 - k_1 p_1 > F_{\text{stick}}
\]


Example 4.7 of Lee and Seshia (2017) \cite{LeSe2017}

Consider a bouncing ball.  

Inelastic collision with ground.  

\[
\Sigma = \lbrace \text{ free } \rbrace
\] 
$(\text{free},\text{free}) $ if $y(t) = 0 $, $\dot{y}(t) = -a\dot{y}(t)$, constant $a$, $0<a<1$. 

Cont. state varialbes of free mode (free state)
\[
\begin{gathered}
	s(t) \equiv X(t)  = \left[ \begin{matrix} y(t) \\ 
\dot{y}(t) \end{matrix} \right] \text{ so } \dot{X}(t) = \left[ \begin{matrix} \dot{y}(t) \\ \ddot{y}(t) \end{matrix} \right] = \left[ \begin{matrix} \dot{y}(t) \\ -g \end{matrix} \right] = \left[ \begin{matrix} 0 & 1 \\ 0 & 0 \end{matrix} \right] X + \left[ \begin{matrix} 0 \\ -g \end{matrix} \right] 
\end{gathered}
\]
\[
\begin{aligned}
	& \ddot{y}(t) = -g \\ 
	& \dot{y}(t) = -gt \\ 
	& y(t) = \frac{-1}{2} gt^2 + h_0  
\end{aligned}
\]





\part{Screws, Bolts, Fasteners (through Shigley's Mechanical Engineering Design)}

cf. Budynas and Nisbett (2014) \cite{BuNi2014}.  

\section{Screws, Fasteners, Design of Nonpermanent Joints}

cf. Ch. 8 "Screws, Fasterners, and the Design of Nonpermanent Joints", Budynas and Nisbett (2014) \cite{BuNi2014}.  

cf. Section 8-1, "Thread Standards and Definitions", pp. 410, Budynas and Nisbett (2014) \cite{BuNi2014}.  
 
pitch $\equiv $ distance between adjacent thread forms, measured parallel to thread axis.  

$N :=$ number of thread forms per unit length (per inch U.S.); pitch $=1/N$  

$d\equiv $ major diameter $:=$ largest diameter of screw thread.  

$d_r \equiv $ minor (or root) diameter $=$ smallest diameter of screw thread.  

$l \equiv $ lead $=$ distance nut moves parallel to screw axis when nut given $1$ turn for \\
single thread $=l=p$ ($p$ is \emph{pitch}) \\
multiple-threaded $- 2 $ or more threads out beside each other (imagine 2 or more strings wound side by side around a pencil).  \\
double-threaded $\to l = 2p$  \\
triple-threaded $\to l=3p$  

Imagine square-threaded power screw, single thread, unrolled or developed, for exactly single turn.  

$d_m:= $ mean thread diameter circle.  $\lambda := $ lead angle of thread.  

Consider sum of all forces \textbf{on} the single screw thread.  

To raise load, force $P_R$ acts to the right \textbf{on} single screw thread.   \\
$\mathbf{P}_R = (P_R,0)$ ; \, $P_R >0$ \\
$\mathbf{N}= N (-\sin{\lambda}, \cos{\lambda} )$  \\ 
friction force $\mathbf{fN} \, \text{ (Shigley) } \equiv - \mu N(\cos{\lambda}, \sin{\lambda})$ \\
$\mathbf{F} =$ sum of all axial forces acting upon normal thread area $=(0,-F)$.  

\begin{equation}
	\mathbf{P}_R + \mathbf{N} + \mathbf{fN} + \mathbf{F} = (P_R-N\sin{\lambda} , - \mu N \cos{\lambda}, N\cos{\lambda} - \mu N\sin{\lambda}, -F)
\end{equation}
Surely $(\mathbf{P}_R + \mathbf{N} + \mathbf{fN} + \mathbf{F})_x =0$, so that $P_R = N( \sin{\lambda} + \mu \cos{\lambda})$.  

We can imagine $m_{\text{thread}} a_{\text{thread}} = N (\cos{\lambda} - \mu \sin{\lambda}) -F = \frac{P_R}{ \sin{\lambda} + \mu \cos{\lambda} }(\cos{\lambda} - \mu\sin{\lambda} ) -F \neq 0 $

Lowering load, $P_L$ acts to the left \textbf{on} simple screw thread.  

$\mathbf{P}_L = (-P_L ,0)$; $\mathbf{fN} \equiv \mu N(\cos{\lambda}, \sin{\lambda})$.  

\begin{equation}
\sum_{\text{ on single screw thread} } \mathbf{F}_i = \mathbf{P}_L + \mathbf{N} + \mathbf{fN} + \mathbf{F} = (-P_L - N\sin{\lambda} + \mu N \cos{\lambda}, N\cos{\lambda} + \mu N \sin{\lambda} - F)
\end{equation}
Surely $(\mathbf{P}_L + \mathbf{N} + \mathbf{fN} + \mathbf{F})_x =0$ so that $P_L = N(\mu \cos{\lambda} - \sin{\lambda} )$.  

We can imagine 
\[
m_{\text{thread}} a_{\text{thread},y} = F_{\text{net},y} = N(\cos{\lambda} + \mu \sin{\lambda}) - F = \frac{ P_L}{ \mu \cos{\lambda} - \sin{\lambda} } (\cos{\lambda} +\mu \sin{\lambda} ) - F
\]

At equilibrium of forces,

For raising load, 
\[
\begin{gathered}
	\mathbf{\tau} = \mathbf{r} \times \mathbf{P} = \mathbf{r} \times \mathbf{P}_R = \mathbf{r} \times \left( F \left( \frac{ \sin{ \lambda } + \mu \cos{\lambda} }{ \cos{ \lambda} - \mu \sin{\lambda}  } \right) \right) = \mathbf{r} \times \left( F \left( \frac{ \tan{ \lambda } + \mu  }{ 1 - \mu \tan{\lambda}  } \right) \right) = \\
= \mathbf{r} \times F \left( \frac{ \frac{ l}{ \pi d_m } + \mu }{ 1 - \mu \frac{l}{\mu d_m } } \right) = \mathbf{r} \times F \left( \frac{ l + \mu d_m \pi }{ \mu d_m - \mu l } \right)
\end{gathered}
\]
For lowering load, 
\[
\begin{gathered}
	\mathbf{\tau} = \mathbf{r} \times \mathbf{P} = \mathbf{r} \times \mathbf{P}_L = \mathbf{r} \times \left( F \left( \frac{  \mu \cos{ \lambda } -  \sin{\lambda} }{ \cos{ \lambda} + \mu \sin{\lambda}  } \right) \right) = \mathbf{r} \times \left( F \left(   \frac{ \mu -  \tan{ \lambda }   }{ 1 + \mu \tan{\lambda}  } \right) \right) = \\
= \mathbf{r} \times F \left( \frac{ \mu -  \frac{ l}{ \pi d_m }  }{ 1 + \mu \frac{l}{\mu d_m } } \right) = \mathbf{r} \times F \left( \frac{ \pi \mu d_m - l }{ \pi d_m + \mu l } \right)
\end{gathered}
\]

Load lowers itself, by causing screw to spin without external effort.  
\begin{equation}
	\mathbf{\tau}_L  = \mathbf{r} \times F \left( \frac{ \pi d_m \mu - l }{ \pi d_m + \mu l } \right)  \leq 0 \text{ or } \mu \leq \tan{\lambda} 
\end{equation}
If $\tau_L >0$, screw is \textbf{self-locking}; in this case, $\mu > \tan{\lambda}$

Efficiency expression (for power screws)
\begin{equation}
\begin{gathered}
	\tau_0(\mu = 0)  = \frac{d_m}{2} F\left( \frac{l }{ \pi d_m} \right)  = \frac{Fl}{2\pi }
\end{gathered}
\end{equation}
Since thread friction has been eliminated, torque required only to raise the load.  

Then define efficiency $\epsilon_{\text{load}}$ as 
\begin{equation}
\begin{gathered}
	\epsilon_{\text{load}} = \frac{ \tau_0 }{ \tau_R} = \frac{Fl}{2\pi \tau_R} = \frac{Fl}{ 2\pi  \frac{d_m}{2} F \left( \frac{ l + \mu d_m \pi }{ \pi d_m - \mu l } \right) } = \frac{l}{ \pi d_m} \left( \frac{ \pi d_m - \mu l }{ l + \mu d_m \pi } \right) = \frac{1 - \frac{ \mu l }{ \pi d_m } }{ 1 + \frac{ \mu d_m \pi }{ l }  } 
\end{gathered}
\end{equation}


\section{Ball Screws; ball screw} 

cf. \href{https://www.thk.com/sites/default/files/documents/uk_pdf/product/general/a/ee_A15.pdf}{THK Ball Screw, General Catalog}

THK provided an excellent overview of the equations/relations (and thus the physics) behind ball screws (THK, \href{https://www.thk.com/sites/default/files/documents/uk_pdf/product/general/a/ee_A15.pdf}{THK Ball Screw, General Catalog}).  

I will include THK's notation and my notation, side by side.  

Let \\
$\beta \equiv \lambda \equiv$ lead angle  \\
$d_p \equiv $ Ball center-to-center diameter \\
$\text{Ph} \equiv l \equiv $ feed screw lead (feed screw is that long piece, the screw; lead is distance but moves in 1 turn) (mm) \\
\begin{equation}
	\tan{\beta} = \frac{l}{\pi d_p} 
\end{equation}

\subsection{Driving torque required to gain thrust} 

Let 
$T \equiv \tau := $ driving torque ($N \cdot mm$) \\
$Fa \equiv f_a := $ frictional resistance on guide surface ($N$) \\
$Fa= \mu \times mg \equiv \mu Mg$ \\
$\mu := $ frictional coefficient of the guide surface  \\
$g := $ gravitational acceleration ($9.8 \, m/s^2$)  \\
$gm \equiv M =$ mass of trasnferred object (kg).  \\
$\eta_1 :=$ positive efficiency of feed screw (see Fig. 1 on A-682 (rotational to linear), from Fig. 1, $0.003 \leq \mu \leq 0.02$ 
\begin{equation}
\tau =  \frac{ f_a l }{ 2\pi \eta_1 }
\end{equation}
$\eta_1$ efficiency is either rotational to linear energy, or force.  I believe it's a ratio between magnitude of forces.  


cf. \href{http://web.mit.edu/2.75/fundamentals/FUNdaMENTALs\%20Book\%20pdf/FUNdaMENTALs\%20Topic\%206.PDF}{FUNdaMENTALS of Design, Topic 6, Power Transmission Elements II, by Alexander Slocum, 2008}.  


cf. \href{https://www.cs.cmu.edu/~rapidproto/mechanisms/chpt2.html}{Introduction to Mechanisms; 2 Mechanisms and Simple Machines}.  

Taking a look at \emph{2 Mechanisms and Simple Machines}, Sec. \emph{2.1.1 Screw Jack}, for Introduction to Mechanisms by Yi Zhang, with Susan Finger and Stephannie Behrens, for \href{https://www.cs.cmu.edu/~rapidproto/home.html}{39-245} Rapid Design through Virtual and Physical Prototyping at Carnegie Mellon University, let \\
$\mathbf{W} :=$ heavy weight to raise $= W(-\mathbf{e}_z)$, \\
$\mathbf{F}:=$ much smaller force applied at handle, \\
$p \equiv l :=$ pitch of screw (distance advanced in 1 complete turn) $=l$, then \\
neglecting friction, \\
force multipled by distance through which it moves in one complete turn $=$ weight lifted times distance through which it's lifted in same time, 

and so by energy conservation
\[
F(2\pi R) = Wp \equiv Wl \text{ or } F = \frac{Wp }{2\pi R}
\]

\subsection{Kinematics}

\subsubsection{Helix Geometry/Mathematics}

Consider a circular helix of radius $a$, slope $\frac{b}{a}$ (or pitch $2\pi b$), described by parametrization 
\[
\begin{aligned}
	& x(t) = a \cos{(t)} \in \mathbb{R} \\ 
		& y(t) = a \sin{(t)} \in \mathbb{R} \\ 
	& z(t) = b(t) \in \mathbb{R}
\end{aligned}
\]
Clearly the radius is $a$ as $x^2 + y^2 = a^2 = r^2$, and the rotation about the (symmetric) axis of rotation (in $z$ direction) is $\theta \equiv \theta(t) = 0$, in \emph{radians}.  

For rotation $\theta$, we travel in the $z$-direction by $z(t) =b\theta$.  So for 1 full rotation (revolution), $\theta = 2\pi$, we have in the $z$-axis by $z(t=2\pi) =2\pi b$.  

In engineering (in general), in particular for screws (lead screws or ball screws), we have the \emph{lead} $l$: \\
$l:= \text{thread lead, i.e. linear distance either screw or nut moves, i.e. (linear) translation length (for 1 full rotation (revolution)) }$.  Clearly
\begin{equation}
l = 2\pi b
\end{equation}

So given input torque $\tau$, efficiency for rotational motion to linear motion (i.e. linear output$/$ rotational input), $\eta_1$, 
\begin{equation}
\begin{gathered}
\eta_1 \tau \theta = Fz \text{ or } \eta_1\tau \theta = F(b\theta) = F \left( \frac{l}{2\pi \theta } \right) \text{ or } \boxed{ F = \frac{2\pi \eta \tau }{l } }
\end{gathered}
\end{equation}

Other sources I looked at: 
Vahid-Araghi and Golnaraghi (2010) \cite{VaGo2010} 



\section{State Machine; Finite State Machine and (finite) graphs; graphs}

cf. Stallings (1983) \cite{Stal1983}

\subsection{Paths}

cf. pp. 553, Section 2. Paths in Stallings (1983) \cite{Stal1983}; pp. 14, Subsection 2.1 Graphs in Serre (1980) \cite{Serr1980}
\begin{definition}[path]
path $p$ in $\Gamma$, of length $n=|p|$, with initial vertex $u$ and terminal vertex $v$, $n$-tuple of edges of $\Gamma$, $p=e_1 e_2 \dots e_n$, s.t. $\forall \, i = 1\dots n-1$, then $\tau(e_i) = i(e_{i+1})$ s.t. $u=i(e_1)$, $v=\tau(e_n)$  

\end{definition}

For $n=0$, given any vertex $v$, $\exists \, !$ unique path $\Lambda_v$, $| \Lambda_v| =0$, s.t. $u=v$, i.e. \\
standard arc of length $n$, $\Lambda_n$, described as interval $[0,n]$ subdivided at integer points, path $p$ is map of graphs $p:A_n \to \Gamma$ s.t. $p(0)=u$, $p(n)=v$.  


\part{Avionics; Embedded Systems; Embedded C}  

\textbf{Introduction to Embedded Systems: Interfacing to the Freescale 9S12} from Valvano was meant to be an introduction (Valvano (2010) \cite{Valv2010}).  Valvano in there suggests \textbf{Embedded Microcomputer Systems} (Valvano (2011) \cite{Valv2011}) for an advanced treatment.  





\end{multicols*}

\begin{thebibliography}{9}


\bibitem{Hugh2000}
Scott B. Hughes.  \textbf{Magnetic braking: Finding the effective length over which the eddy currents form}.  
\href{https://drive.google.com/file/d/0Bwo3W0v5P04LX29XT2NFeVY0a1E/view}{Magnetic braking: Finding the effective length over which the eddy currents form}  


\bibitem{Smyt1968}
William R. Smythe, \textbf{Static and Dynamic Electricity}.  3rd ed. (McGraw-Hill, New York, 1968).  

\bibitem{Jack1998}
J.D. Jackson.  \textbf{Classical Electrodynamics} Third Edition.  Wiley.  1998.   ISBN-13: 978-0471309321

\bibitem{Herm1974}
Robert Hermann.  \textbf{Linear Systems Theory \& Introductory Algebraic Geometry (Interdisciplinary Mathematics Series)}.  Math Science Pr (June 1974)
ISBN-13: 978-0915692071

\bibitem{BuNi2014}
Richard Budynas, Keith Nisbett.  \textbf{Shigley's Mechanical Engineering Design} (McGraw-Hill Series in Mechanical Engineering) 10th Edition.  McGraw-Hill Series in Mechanical Engineering.  ISBN-13: 978-0073398204

EY : 20170606 -  I was only able to obtain a copy of the 9th edition to use.  If you'd like to help us keep up to date with the 10th edition or to help the rLoop in general, please donate to the \href{rLoop}{www.rLoop.org}!

\bibitem{VaGo2010}
Orang Vahid-Araghi, Farid Golnaraghi.  \textbf{Friction-Induced Vibration in Lead-Screw Drives}.  Springer.  2010.  e-ISBN 978-1-4419-1752-2



\bibitem{Stal1983}
John R. Stallings.  "Topology of finite graphs."  \textbf{Invent. math. 71}, 551-565 (1983)  \textbf{Inventiones mathematicae}.  Springer-Verlag 1983.  

\bibitem{Serr1980}  

Trees (Springer Monographs in Mathematics) 1st ed. 1980. Corr. 2nd printing 2002 Edition
by Jean-Pierre Serre (Author), J. Stilwell (Translator) 

\bibitem{Valv2010}  
Jonathan W. Valvano.  \textbf{Introduction to Embedded Systems: Interfacing to the Freescale 9S12}.  1st. Ed. 2010.  Cengage Learning.  USA.  

\bibitem{Valv2011}
Jonathan W. Valvano.  \textbf{Embedded Microcomputer Systems: Real Time Interfacing}.  3rd. Ed. 2011.  Cengage Learning.  USA.  

\bibitem{LeSe2017}
%Edward Ashford Lee and Sanjit Arunkumar Seshia.  \textbf{Introduction to Embedded Systems} \textbf{
Edward A. Lee and Sanjit A. Seshia, \textbf{\href{http://leeseshia.org/index.html}{Introduction to Embedded Systems}}, A Cyber-Physical Systems Approach, Second Edition, MIT Press, ISBN 978-0-262-53381-2, 2017. 

\end{thebibliography}

\end{document}
